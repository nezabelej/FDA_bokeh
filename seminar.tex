\documentclass[a4paper,10pt]{article}
\usepackage[utf8]{inputenc}
\usepackage{graphicx}
\usepackage{float}

\title{Analiza podatkov Food And Drug Administration z orodjem Bokeh}
\author{Neža Belej (63120340), Matej Dolenc (63120178)}

\begin{document}

\maketitle
\section {Bokeh}
Bokeh je interaktivna knjižnica programskega jezika Python. Omogoča elegantno in interaktivno vizualizacijo nad veliko množico podatkov.
Arhitektura Bokeha sestoji iz dveh delov: izdelava grafov z programskem jeziku Python in pa izris v brskalniku s knjižnico BokehJS. Grafi v Pythonu se pretvorijo v JSON format, saj to zahteva BokehJS. Takšen dizajn je zelo fleksibilen, saj omogoča, da delo tudi z drugimi programskimi jeziki (R, Scala, Lua,... ) lahko privede do enakih Bokeh grafov v brskalniku. \\
Če želimo sinhronizacijo Bokeh grafov in interaktivne vizualizacije v brskalniku, moramo uporabiti tudi strežnik Bokeh. Tako imamo omogočeno avtomatično posodabljanje uporabniškega vmesnika v brskalniku glede na naše klike in vnose.

\subsection{Težave}
Med spoznavanjem orodja Bokeh smo opazili, da ima orodje še kar nekaj hroščev. V okviru izdelave smo velikokrat naleteli na repozitorij na Githubu, kjer je trenutno kar 764 odprtih nalog ("issues"): \\ https://github.com/bokeh/bokeh/issues . \\Orodje je preprosto za uporabo, ima dobro dokumentacijo, vendar ima še veliko lukenj. Primer: \\
Ob prikazu grafa najpogostejših reakcij smo želeli na interaktiven način izvesti prikaz reakcij v različnih časovnih obdobjih.
Zato smo najprej želeli uporabiti element DatePicker, kjer bi lahko izbrali začetni in končni datum. Opazili smo, da ima element v trenutni fazi zelo slab izgled \\  (issue: https://github.com/bokeh/bokeh/issues/4503). Zato smo poizkusili z uporabo elementa DateRangeSlider, ki naj bi imel na različnih straneh drsnika začetni in končni datum. Element se ni prikazal \\ (issue: https://github.com/bokeh/bokeh/issues/2268 ). Zato smo bili primorani uporabiti dva navadna drsnika, enega za začetno, drugo za končno leto. 

\section{OpenFDA}
OpenFDA (Food and Drug Administration) nam na svojih straneh omogoča dostop do 100 GB velike množice podatkov, kjer lahko poizvedujemo o medicinskih poročilih o zdravilih in hrani; na primer stranski učinki zdravil ali odpoklic prehrambenih produktov. OpenFDA je namenjena predvsem za poizvedbe preko njihovega zmogljivega API-ja, ki ima v ozadju implementiran učinkovit Elastic Search. Ta nam omogoča hitro in preprosto poizvedovanje po podatkih. \\
Podatki, namenjeni prenosu, so razbiti na veliko število datotek v JSON formatu. Če želimo prenesti podatke, moramo paziti, da ob vsaki posodobitvi podatkov znova prenesemo celotno zbirko podatkov. Podatki so v dokumentni, nenormalizirani obliki, kar omogoča hitro iskanje.
\section{Navodila za izvajanje}
Za zagon projekta je potrebno imeti nameščeno orodje Anaconda. Nato iz konzole Bash ali pa Windows-ovega CMD-ja zaženemo ukaz: \\ \textit{conda install bokeh} \\
Nato se premaknemo v direktorij, ki vsebuje main.py našega projekta in natipkamo ukaz: \\ \textit{bokeh serve .} \\ V brskalniku se pomaknemo na \textit{localhost:5006}. 5006 tukaj predstavlja številko vrat (port), ki se nam ob zagonu strežnika izpiše v konzoli. 
\section{Analiza podatkov}
\subsection{prvi}

%\begin{figure}[H]
 % \caption{Input image.}
  %\centering
    %\includegraphics[width=0.7\textwidth]{0099.png}
%\end{figure}

\section{Reference}
http://bokeh.pydata.org/en/latest/ \\
https://github.com/bokeh/bokeh/ \\
https://open.fda.gov/

\end{document}

